In a game where pure strategy is used Nash Equilibrium may or may not exist. 
 In the game of 
 \ref{tab:table1}
 prisoner's dilemma
 there is only one Nash Equilibirum, (Confess,Confess), i.e. both players will confess.

This can be proved as follows. When, P1 confesses, P2 is better off confessing than denying because he gets 5 years confessing which is less than 10 years

if he denies. When, P2 confesses, P1 is better off confessing than denying, because  P1 gets 5 years when he confesses but he gets 10 years on denying. The above analysis, shows that (C,C) represents Nash Equilibrium. Similarly, one can show that this is the only Nash Equilibrium in this game.
By examining the four possible pairs of actions in this game, one can see that the action pair (Confess,Confess) is a pure strategy Nash Equilibrium because if player 2 chooses Confess,player 1 is better off choosing Confess than Deny. Similarly given that player 1 chooses to Confess, player 2 is better off choosing Confess than Deny.