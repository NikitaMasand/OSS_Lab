Two players choose independently either Head or Tail and report it to a central authority. If both choose the same side of the coin , player 1 wins, otherwise 2 wins.

A game has the following :-
\begin{enumerate}
    \item \textbf{Set of Players:}             
   The two players  who are choosing either Head or Tail in the Coin Matching Game form the set of players i.e. P={P1,P2}
   \item  \textbf{Set of Rules:}
    There are ceratin rules which each player has to follow while playing the game. Each player can safely assume that others are following these rules. In coin matching game each player can choose either Head or Tail. He has to act independently and made his selection only once. Player 1 wins if both selections are the same othrwise player 2 wins. These form the Rule set R for the Coin Matching Game.
    \item   \textbf{Set of Strategies :}
    For example in Matching coins S1 = { H, T}  and S2 = {H,T}  are the strategies of the two players. Which means each of them can choose either Head or Tail. 
    \item\textbf{Set of Outcomes: }
     In matching Coins its {Loss, Win} for both players.
     $\mathrm { S } _ { 1 } \mathrm { x } \mathrm { S } _ { 2 } = \{ ( \mathrm { H } , \mathrm { H } ) , ( \mathrm { H } , \mathrm { T } ) , ( \mathrm { T } , \mathrm { H } ) , ( \mathrm { T } , \mathrm { T } ) \}$
     \item \textbf{Payoff:}
    This is the amount of benefit a player derives if a particular outcome happens. In general its different for different players. 
    Let the payoffs in  Coin Matching Game be, \\
    \begin{aligned} $\mathrm { u } _ { 1 } ( \mathrm { Win } ) & = 100 \\
    \mathrm { u } _ { 1 } ( \mathrm { L } \mathrm { oss } ) & = 0$ \end{aligned} \\ \\
    \begin{aligned} $\mathrm { u } _ { 2 } ( \mathrm { Win } ) & = 100 \\
    \mathrm { u } _ { 2 } ( \mathrm { L } \mathrm { oss } ) & = 0$ \end{aligned} \\ \\
    Both the players would like to maximize their payoffs (rationality) so both will try to win. Now lets consider a slightly different case. We redefine the payoffs as,
Player 1 is competitor so \\
\begin{aligned} $\mathrm { u } _ { 2 } ( \mathrm { Win } ) & = 100 \\
    \mathrm { u } _ { 2 } ( \mathrm { L } \mathrm { oss } ) & = 0$ \end{aligned} \\ \\
    While player 2 is a very concerned about seeing player 1 happy (player 1 is his little brother) so for him \\
    
    \begin{aligned} $\mathrm { u } _ { 2 } ( \mathrm { Win } ) & = 10 \\
    \mathrm { u } _ { 2 } ( \mathrm { L } \mathrm { oss } ) & = 100$ \end{aligned} \\ \\
    In this situation only player 1 would try hard to win while player 2 will try to lose. The point to note is that each player tries maximize his payoff for which he/she would like to get the Outcome which gives him maximum payoff.
 

    Informally we can say the players sit across a table and play the game according to the set of rules. There is an outcome for each player when the game  ends. each player derives a pay off from this outcome. For example an outcome of victory brings payoff in terms of awards and fame to the cricket players, while loss means no payoff. Because all the players are rational beings they will try to maximize their payoffs. In non co-operative games players don't know what other players are doing. So they have to make the moves without looking at what others are doing. 
      Each player chooses a strategy i.e. set of moves he would play . \\
      \textbf{Strategy:}
    It is the set of moves that a player would play in a game. Being rational a player would chose the startegy in such a way as to maximize his/her payoff.
\end{enumerate}