Game theory has found its applications in numerous fields such as Economics, Social Science, Political Science, Evolutionary Biology. Game theory is now finding its applications in computer science. The nature of computing is changing because of success of Internet and the revolution in Information technology. The advancement in technologies have made it possible to commodities the components such as network, computing, storage and software. In the new paradigm, there are multiple entities (hardware, software agents, protocols etc.) that work on behalf of different autonomous bodies (such as a user, a business etc.) and provide services to other similar entities. Internet has made is possible for many such geographically distributed antonymous entities to interact with each other and provide various services. These entities will work  for their respective owners to achieve their individual goals (maximize their individual payoffs), as opposed to obtaining a system optima (that is socially desirable). This results in an entirely different paradigm of computing where the "work" is performed in a completely distributed/decentralized fashion by different entities where the primary objective of each entity  is to maximize the objective of its owner. Therefore, it is important to study traditional computer science concepts such as algorithm design, protocols, performance optimization under a game-theoretic model.  