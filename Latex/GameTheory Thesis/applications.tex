\textbf{\large Cloud Computing}

In cloud computing, game theory is used for modeling complex interactions between cloud providers--whose aim is to minimize cost while maximizing resource utilization--on one hand and  a number of service providers often with conflicting objectives of maximizing Quality of Service at minimal cost. In such a situation, a game is set up based on an utility function that will eventually steer game play towards an equilibrium state, the Nash Equilibrium, where no players could receive an incentive to change their strategy--that state where objectives of all players are balanced. There are many scenarios in cloud resource management involving spot pricing of cloud resource addressed by auction/bidding games. \\
\textbf{\large Network Security}
A similar case applies in network security. Recently researchers have been developing both deterministic and stochastic security games to study security problems  as an optimization decision problem comprising multiple players notably the attackers and  the  defenders. \\
\textbf{\large Resource Allocation and Networking}
Computer network bandwidth can be viewed as a limited resource. The users on the network compete for that resource. Their competition can be simulated using game theory models. No centralized regulation of network usage is possible because of the diverse ownership of network resources.
The problem is of ensuring the fair sharing of network resources. For example, ten Stanford students on the same local network need access to the Internet. Each person, by using their network connection, diminishes the quality of the connection for the other users. This particular case is that of a volunteer's dilemma. That is, if one person abstains from using the network, the other people will be better off, but that person will be worse off.

If a centralized system could be developed which would govern the use of the shared resources, each person would get an assigned network usage time or bandwidth, thereby limiting each person's usage of network resources to his or her fair share.

As of yet, however, such a system remains an impossibility, making the situation of sharing network resources a competitive game between the users of the network and decreasing everyone's utility. \\
\textbf{\large Artificial Intelligence: }
One of the marks that differentiates a human from a machine is the human's ability to make independent decisions based on environmental stimuli. Most computer programs that are required to make any sort of a decision are currently pre-programmed with the lists of decisions based on a number of conditions. However, if those conditions are not met in some way or are altered, computers have no way of making decisions they were not programmed to make.
In the future, AI programs may be endowed with the ability to make new decisions unplanned for by their creators. This would require the programs to be able to generate new payoff matrices based on the observed stimuli and experience. A program that is able to do that would be capable of learning and would, in a lot of ways, resemble the human decision-making process.