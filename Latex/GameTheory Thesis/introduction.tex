Game Theory can be regarded as a multi-agent decision problem. Which means there are many people contending for limited rewards/payoffs. They have to make certain moves on which their payoff depends. These people have to follow certain rules  while making these moves.  Each player is supposed to behave rationally.\\
\textbf{Rationality:} In the language of Game Theory rationality implies that each player tries to maximize his/her payoff irrespective to what other players are doing.
In essence each player has to decide a set of moves which are in accordance with the rules of the game and which maximize his/her rewards.\\
Game Theory can be classified in two branches
\begin{enumerate}
    \item Non co-operative game theory :  In this case the players work independently without assuming anything about what other players are doing.
    \item Co-operative game theory: Here players may co-operate with one another.
\end{enumerate}
Game Theory has found applications in Economic, Evolutionary Biology, Sociology, Political Science etc, now Its finding applcations in Computer Science. 

